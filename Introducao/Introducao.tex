\section{Motivação e Justificativa}

	Um trocador de calor pode ser definido como um dispositivo em que ocorre uma transferência de calor entre duas substâncias que estejam em temperaturas distintas. Geralmente, as substâncias envolvidas são fluidos. Os trocadores de calor podem ser classificados com relação a diversos critérios, como por exemplo o modo de troca de calor, tipo de construção, entre outros. \cite{kreith2011}
	
	
	
\section{Objetivos}
	De acordo com as informações expostas acima, este Projeto Final de Curso (PFC) possui o seguinte objetivo geral:
	
	\begin{itemize}
		\item 
		Propor e testar uma solução baseada em sistemas embarcados para monitoramento e operação remota de um trocador de calor;
	\end{itemize}
	
	Além do objetivo geral, o projeto possui também os seguintes objetivos específicos:
	
	\begin{itemize}
		\item 
		Comparar a solução embarcada com um software de prateleira tradicional;
		\item 
		Proporcionar uma base de conhecimento para implementações de sistemas semelhantes para outras plantas do laboratório;
		\item 
		Possibilitar a inserção de mais funcionalidades ao sistema, como por exemplo um algoritmo de autovalidação da planta;
	\end{itemize}
	

\section{Local de Realização}
	O projeto foi desenvolvido no Laboratório de Operações e Processos Industriais, um dos laboratórios que pertencem ao Departamento de Engenharia Química (DEQ) da UFMG. Estes laboratórios são utilizados para fins didáticos e de pequisa.

\section{Estrutura da Monografia}